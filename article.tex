\documentclass[11pt]{article}
\usepackage{amsmath}
\usepackage{color}
\usepackage{cite}
\title{Arithmatic and Geometric Progression}
\author{Aayush Ojha}
\date{\today}
\begin{document}
\maketitle
\section{ARITHMATIC PROGRESSION}
In mathematics, an arithmetic progression (AP) or arithmetic sequence is a sequence of numbers such that the difference between the consecutive terms is constant. For instance, the sequence $5$, $7$, $9$, $11$, $13$, $15$ $\ldots$ is an arithmetic progression with common difference of $2$.\\
If the initial term of an arithmetic progression is $a_{1}$ and the common difference of successive members is $d$, then the $nth$ term of the sequence ($a_{n}$) is given by:\\
\\
$$ a_{n} = a_{1} + (n-1)d,$$\\
\\
and in general\\
\\
$$ a_{n} = a_{m} + (n - m)d.$$

\subsection{SUM OF AP}
The sum of the members of a finite arithmetic progression is called an arithmetic series. For example, consider the sum:

   $$2 + 5 + 8 + 11 + 14$$

This sum can be found quickly by taking the number n of terms being added (here 5), multiplying by the sum of the first and last number in the progression (here 2 + 14 = 16), and dividing by 2:

$$\frac{n(a_{1} + a_{n})}{2}$$

\subsection{DERIVATION}
To derive the above formula, begin by expressing the arithmetic series in two different ways \cite{Arith}:

$$ S_{n}=a_{1}+(a_{1}+d)+(a_{1}+2d)+\ldots+(a_{1}+(n-2)d)+(a_{1}+(n-1)d)$$

   $$ S_{n}=(a_{n}-(n-1)d)+(a_{n}-(n-2)d)+\ldots+(a_{n}-2d)+(a_{n}-d)+a_{n}.$$

Adding both sides of the two equations, all terms involving d cancel:

    $$2S_{n}=n(a_{1} + a_{n}).$$

Dividing both sides by 2 produces a common form of the equation:

    $$S_{n}=\frac{n}{2}( a_{1} + a_{n}).$$

An alternate form results from re-inserting the substitution $a_{n} = a_{1} + (n-1)d$:

    $$S_{n}=\frac{n}{2}[ 2a_{1} + (n-1)d].$$

Furthermore the mean value of the series can be calculated via: $S_{n} / n$:

    $$\overline{n} =\frac{a_{1} + a_{n}}{2}.$$ 
\section{GEOMETRIC PROGRESSION}
In mathematics, a geometric progression, also known as a geometric sequence, is a sequence of numbers where each term after the first is found by multiplying the previous one by a fixed, non-zero number called the common ratio. For example, the sequence $2$, $6$, $18$, $54$, $\ldots$ is a geometric progression with common ratio $3$. Similarly $10$, $5$, $2.5$, $1.25$, $\ldots$ is a geometric sequence with common ratio $\frac{1}{2}$.

Examples of a geometric sequence are powers $r^{k}$ of a fixed number $r$, such as $2^{k}$ and $3^{k}$. The general form of a geometric sequence is\\
\\
$$a,ar,ar^{2},ar^{3},ar^{4}, \ldots$$
\\
where $r \neq 0$ is the common ratio and $a$ is a scale factor, equal to the sequence's start value.
\subsection{SUM OF GP}
A geometric series is the sum of the numbers in a geometric progression. For example:

   $$2 + 10 + 50 + 250 = 2 + 2 \times 5 + 2 \times 5^{2} + 2 \times 5^{3} ,$$

Letting a be the first term (here $2$), m be the number of terms (here $4$), and $r$ be the constant that each term is multiplied by to get the next term (here $5$), the sum is given by:

$$\frac{a(1-r^{m})}{1-r}$$

In the example above, this gives:

   $$2 + 10 + 50 + 250 = \frac{2(1-5^4)}{1-5} = \frac{-1248}{-4} = 312.$$

\subsection{DERIVATION OF SUM OF GP}
To derive this formula, first write a general geometric series as \cite{geo}:

$$\sum_{k=1}^{n} ar^{k-1} = a+ar +ar^{2}+ar^{3}+\ldots+ar^{n-1}.$$

We can find a simpler formula for this sum by multiplying both sides of the above equation by $1-r$, and we'll see that
\begin{equation*}
\begin{aligned} (1-r) \sum_{k=1}^{n} ar^{k-1} &{} = (1-r)(a + ar +ar^{2}+ar^{3}+\ldots+ar^{n-1}) \\ &{} = a + ar +ar^{2}+ar^{3}+\ldots+ar^{n-1} \\ &{}      - ar -ar^{2}-ar^{3}-\ldots-ar^{n-1} - ar^{n} \\ &{}= a - ar^{n} \end{aligned}
\end{equation*}
since all the other terms cancel. If $r \neq 1$, we can rearrange the above to get the convenient formula for a geometric series that computes the sum of n terms:

   $$ \sum_{k=1}^{n} ar^{k-1} = \frac{a(1-r^n)}{1-r}.$$ 
 \bibliographystyle{plain}
 \bibliography{references}
\end{document}
